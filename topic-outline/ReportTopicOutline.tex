\documentclass{homework}

\title{Report Outline for \\Creamy and Delicious Chocolate Fudge}
\author{J. Lefebvre\\\small{jayden.lefebvre@mail.utoronto.ca}\\\small{UTorID: lefebv69}}

\begin{document}

\maketitle

\section{Polymer for Latching Compliant Mechanisms}

\subsection{Introduction}

A "compliant mechanism" is a subset of simple mechanical devices used to transform force and motion via plastic deformation and deflection (thus, complying to the stimuli imparted onto it).
Compliant mechanisms have become increasingly popular as an alternative to more traditional rigid systems for a variety of reasons, mostly dependent on the use case [1].
Their potential use cases are immensely broad, ranging from sensitive surgical instrumentation (that must respond and react to subtle muscle movements and deflect so as to not cause tissue damage) and 

\subsection{Property Problem, Microstructure Solution}



\subsubsection{Critical Use Cases}

Nuclear fallback safety systems

\subsubsection{Consistent Path Tracing}



\subsubsection{State Latching}

"Latches" are component sub-mechanisms built in to larger compliant mechanisms used to "save" the positional state of the mechanics in the compliant mechanism, in a similar way to how a transistor-based latch saves the binary logic-level state of a digital system.

\subsection{Conclusion}

\section{Alloy for Low-Load, Long-Term Load Cells}

\subsection{Introduction}
\subsection{Property Problem, Microstructure Solution}
\subsection{Conclusion}

\section{Polymer Alternative to PETG for 3D Printing}

\subsection{Introduction}
\subsection{Property Problem, Microstructure Solution}
\subsection{Conclusion}

\section{References}

\begin{enumerate}
    \item "Compliant Mechanisms Explained", https://www.compliantmechanisms.byu.edu/about-compliant-mechanisms
\end{enumerate}

\end{document}